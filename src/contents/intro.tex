%! Author = charon
%! Date = 1/2/23

\section{Einführung}\label{sec:einfuhrung}
% kurz erklären und einführen wieso die begriffe wichtig sind und hinleiten zum kernthema log4j
%%erklären wieso manche sachen nur briefly behandeln
\gls{iot}Geräte finden in der heutigen Zeit immer mehr Verwendung im Alltag der Menschheit.
Sei es die Smart-Fridge, welche einem ihre aktuelle Innentemperatur per App mitteilt oder die neue Philips hue smart light bulb,
welche sich dem Verbrauchsverhalten des Nutzers anpasst bis hin zu einem Automatisierungssensor zum Kalkulieren der Auslastung eines Laufbandes in einem großen Unternehmen.
Auch die ~\gls{sv} der Hochschule Hof hat sich diese Möglichkeit nicht entgehen lassen sich den Alltag eines Events einfacher zu gestalten.
Zum komfortableren Bewältigen der Events, wie die BOOM-Party oder diverse Semester Opening feiern gehört eine gute Koordination untereinander.
Diese Koordination wurde durch die Organisationseinheit Technik der ~\gls{sv} mittels einer Webapplikation, welche einen Echtzeit-Chat mithilfe der Telegram ~\gls{api} überträgt, auf einem Raspberry Pi 3 umgesetzt.
Das ~\gls{iot}-Gerät ist mit einem Raspbian ~\gls{os} eingerichtet und besitzt einen SSH Server, welcher es Entwicklern und Maintainern der Web-App ermöglicht diese weiterzuentwickeln und zu warten.
Um eine genügende Sicherheit des Geräts zu gewährleisten und sensible Daten der Hochschule und auch der ~\gls{sv} nicht in Gefahr zu bringen, soll der Raspberry Pi 3 mit seinen installierten und verwendeten Technologien genügend
abgesichert werden.\\
\blankline
Die folgenden Kapitel beschäftigen sich mit der Problematik ein ~\gls{iot}-Gerät ordnungsgemäß und angemessen abzusichern.
Zuerst wird die initiale Konfiguration genauer angesehen.
Sie beinhaltet das Konfigurieren eines Benutzers mit wenig Privilegien, welcher zum Login von außerhalb über ~\gls{ssh} verfügbar sein soll.
Ebenfalls beschäftigt sich das Kapitel mit dem Deaktivieren des default Nutzers.
Als Nächstes wird das Einrichten von automatischen Sicherheitsupdates veranschaulicht.
Darauf folgt die Konfiguration einer klein gehaltenen Firewall, welche es ermöglicht unerwarteten Netzwerkverkehr anhand des Whitelisting-Prinzips zu blockieren.\\
Im Anschluss wird der von der Web-Applikation verwendete Nginx Reverse Proxy konfiguriert, um ~\gls{DoS} attacken stück weit vorzubeugen und Nginx nur die Berechtigungen zu erteilen, welche es braucht. \\
Der nächste Punkt geht auf bekannte und oft ausgenutzte Schwachstellen ein und wie man diese mittels weiterer Techniken mitegieren kann. \\
Zum Schluss wird die Arbeit in ihrer Gesamtheit evaluiert und ein Fazit gezogen.