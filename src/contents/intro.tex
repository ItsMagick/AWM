%! Author = charon
%! Date = 1/2/23

\section{Einführung}\label{sec:einfuhrung}
% kurz erklären und einführen wieso die begriffe wichtig sind und hinleiten zum kernthema log4j
%%erklären wieso manche sachen nur briefly behandeln
Diese Arbeit beschäftigt sich mit dem \gls{cve}-2021-44228, auch bekannt als \textit{Log4Shell}.
Bei dieser Schwachstelle handelt es sich um einen Zero-Day-Exploit.

Unter einem Zero-Day-Exploit versteht man ``[die] Ausnutzung einer Schwachstelle, die nur dem Entdecker bekannt ist [\ldots].''\footcite{bsizeroday}
Die Schwachstelle wurde am 09. Dezember 2021 entdeckt.
Betroffen hierbei ist die oft eingesetzte Bibliothek Log4j (Version 2).\footcite{lunasec}

In den folgenden drei Kapiteln werden zuerst grundlegende Begriffe eingeführt, die zum Verständnis im weiteren Verlauf der Arbeit wichtig sind.
Zudem werden in dieser Arbeit einige Technologien, wie beispielsweise die tatsächliche Implementierung des Loggers oder die Serialisierung der Objekte in der \gls{jvm}, grob umrissen.

\input{contents/log4j}
\input{contents/jndi}
\input{contents/ldap}
