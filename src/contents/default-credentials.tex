%! Author = chaorn
%! Date = 13.02.24

\section{Initiale Konfiguration}\label{sec:initiale-konfiguration}
\subsection{Benutzerverwaltung}\label{subsec: benutzerverwaltung}
Zur initialen Konfiguration gehört das Verändern der default Nutzer, um den Einfallswinkel für einfache Angriffe zu minimieren.
Um diesem Fehler vorzubeugen wird zunächst ein neuer Nutzer angelegt.
Der Nutzer in diesem Beispiel trägt den Namen \textit{svuser}. \\
%! Author = Charon
%! Date = 14/02/2024

\begin{lstlisting}[language=bash,caption=bash Befehl zum erzeugen eines neuen Benutzers,label=adduser]
    $ sudo adduser svuser
\end{lstlisting}
Nun wird ein zweiter Nutzer \textit{svadmin} angelegt.
Dieser ist gerade für die Wartung und Weiterentwicklung des Raspberry Pis wichtig, um weitere Änderungen am Betriebssystem vornehmen zu können.
Im Folgenden werden alle mit \texttt{sudo} ausgeführten Befehle über diesen Benutzer durchgeführt.
Um dem Nutzer die entsprechenden Rechte zu geben, muss er zuerst zur \texttt{sudoers} gruppe hinzugefügt werden.
\input{listings/hinzufügen-sudo}
\\
\blankline
Mit dem neu erzeugten Nutzer sollen weitere Änderungen vorgenommen werden.
Dazu gehört das Abschalten des default Nutzers \textit{pi} mit hohen Privilegien. \\
%! Author = Charon
%! Date = 14/02/2024

\begin{lstlisting}[language=term,caption=Abschalten des default Nutzers,label=block-default]
    $ sudo usermod --lock --exiredate 1 pi
\end{lstlisting}
\blankline
Im Anschluss ist es eine gute Idee den \texttt{root} Benutzer zu konfigurieren.
Hierzu wird der~\gls{ssh}-Zugriff von diesem Benutzer abgeschaltet.
Um das zu bewerkstelligen, wird eine Benutzergruppe \texttt{ssh-users} angelegt und der aktuell angemeldete Benutzer dieser Gruppe hinzugefügt.
Zum Inkrafttreten des Verbots, sich mit dem \texttt{root} Nutzer anmelden zu können, müssen folgende Änderungen in der~\gls{ssh}-Konfigurationsdatei in \texttt{/etc/ssh/sshd\_config}\label{ssh-config-file} vorgenommen werden:
\begin{itemize}
    \item PermitRootLogin no
    \item AllowGroups ssh-users
\end{itemize}
\begin{lstlisting}[language=term,caption=Ändern des Passworts,label=passwd]
    $ sudo groupadd ssh-users
    $ sudo usermod -a -G ssh-users $USER
\end{lstlisting}
Darauf folgt für das Inkrafttreten der Änderungen den~\gls{ssh}-Prozess neu zu starten.
%! Author = chaorn
%! Date = 16.02.24
\begin{lstlisting}[language=term, caption=Neustart des ssh Prozesses,label=restart-ssh]
    $ sudo systemctl restart ssh
\end{lstlisting}


\subsection{Abschalten der Wireless Interfaces}\label{subsec:abschalten-des-wireless-interface}
Ein weiterer präemptiver Schritt zum Vorbeugen von weitverbreiteten Angriffsvektoren besteht daraus das \textit{Wireless Interface} des Raspberry Pis zu deaktivieren.
Diese Entscheidung beruht auf der Tatsache, dass der Pi3 mit einer statischen~\gls{ip}-Adresse versehen wurde und nur über ~\gls{lan} erreichbar sein soll, um die Überwachung
des Netzwerktraffic über den IT-Service einfacher zu gestalten. \\
%! Author = Charon
%! Date = 14/02/2024
\begin{lstlisting}[language=term, caption= Deaktivieren der Wireless Interfaces,label={lst:disable-wlan}]
    $ sudo vim /boot/config.txt
    ....
    # Disable wireless services
    dtoverlay=pi3-disable-wifi
    dtoverlay=pi3-disable-bt
\end{lstlisting}

Hierzu wurde auch das Bluetooth Interface deaktiviert, da es derzeit dieser Funktionalität nicht bedarf.
Nach dem Anfügen der Konfigurationsdaten in der \texttt{config.txt} folgt ein Neustart des Systems, um überprüfen zu können, ob die Veränderungen übernommen wurden.
\subsection{optional: SSH Konfiguration}\label{subsec:ssh-konfiguration}
Zum weiteren Absichern des~\gls{ssh}-Zugangs wäre es möglich dem erstellten Nutzer unter \texttt{/home/svuser/.ssh/authorized\_keys} einen Public-Key abzulegen.
Dieser könnte dann zum passwortlosen Einloggen mittels des~\gls{ssh}-Keys verwendet werden.
Um keine Redundante Einloggmöglichkeiten zu ermöglichen, werden in der in~\ref{ssh-config-file} beschriebenen Konfigurationsdatei die Zeilen
\begin{itemize}
    \item PubkeyAuthentication yes
    \item PasswordAuthentication no
\end{itemize}
hinzugefügt.
Die Änderung tritt jedoch erst in Kraft, wenn der im Hintergrund laufende~\gls{ssh} Prozess neu gestartet~\ref{lst:restart-ssh} wird.
Von dieser Konfiguration wurde in diesem Projekt aufgrund Inkonsistenz der Besetzung der~\gls{it}-Abteilung der~\gls{sv} abgesehen.
