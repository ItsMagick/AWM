%! Author = chaorn
%! Date = 13.02.24

\section{Verbreitete Schwachstellen und ihre Mitigationen}\label{sec:bekannte-schwachstellen-und-ihre-mitegationen}
Zur weiteren Verstärkung der Credentials für den~\gls{ssh}-Nutzer \texttt{svuser} und, um einer Brute-Force-Attacke vorzubeugen~\cite{rasp-vuln-prev},
wird in dieser Sektion die Installation und Einrichtung von fail2ban beschrieben.
Fail2ban~\cite{fail2ban} ist eine Open-Source-Software, welche Logdateien scant und anhand der darin enthaltenen~\gls{ip}-Adresse diese für einen konfigurierbaren
Zeitraum bannt. \\
Installiert werden kann diese Software entweder über das offizielle Github Repository oder für auf Debian basierten Systemen (oder ähnliche) über den im
Betriebssystem verfügbaren Paket-Manager.
%! Author = chaorn
%! Date = 16.02.24

\begin{lstlisting}[language=term,caption=Installation von fail2ban,label=installation-fail2ban]
    $ sudo apt install fail2ban
\end{lstlisting}

Die Installation beinhaltet das Einrichten eines Hintergrundservice, welcher jedoch initial nicht aktiv ist.
In der bereits vorkonfigurierten Datei \texttt{/etc/fail2ban/jail.conf} existieren Konfigurationen, wie
eine Maximalanzahl für Versuche zum Login über~\gls{ssh} oder Einstellungen für zahlreiche andere Webframeworks.
Für bessere Wartbarkeit der Software wird hier eine \texttt{jail.local} Datei angelegt, da bei jedem Update von fail2ban die Konfigurationsdatei automatisch mit einer
neueren Version überschrieben wird.
%! Author = chaorn
%! Date = 16.02.24

\begin{lstlisting}[language=term,caption=Anlegen einer lokalen jail.conf Datei,label={lst:fail2ban-local}]
    $ sudo cp /etc/fail2ban/jail.conf /etc/fail2ban/jail.local
\end{lstlisting}
\blankline
In der Konfigurationsdatei gibt es einen Eintrag mit dem Namen \texttt{bantime}.
Für viele Brute-Force-Attacken spielt Zeit eine große Rolle.
Um einem Angreifer in diesem Szenario \"den Spaß zu verderben\" wird mit dem Setzen der bantime, nach mehrmaligem Fehlversuch, auf 10 Minuten mit \texttt{bantime=10m}
bereits viel geholfen.
Weitere Konfigurationen können in folgender Form in der \textbf{jail.local} Datei vorgenommen werden.\\
%! Author = chaorn
%! Date = 16.02.24

\begin{lstlisting}[language=term,caption=Konfiguration des ssh jails von fail2ban,label=fail2ban-ssh]
    $ sudo nano /etc/fail2ban/jail.local
        [sshd]
        enabled = true
        port = ssh
        filter = sshd
        logpath = /var/log/auth.log
        maxretry = 3
        bantime = 10m
\end{lstlisting}
Mithilfe dieser Konfiguration ist es nun nicht mehr möglich einen Brute-Force-Angriff auf den \gls{ssh}-Port durchzuführen.
Zum Inkrafttreten der Änderungen und zur Aktivierung des~\gls{ssh}-jails muss der fail2ban Service gestartet werden.
%! Author = chaorn
%! Date = 16.02.24
\begin{lstlisting}[language=term,caption=Starten von fail2ban,label=fail2ban-ssh]
    $ sudo systemctl enable fail2ban
    $ sudo systemctl start fail2ban
\end{lstlisting}