%! Author = chaorn
%! Date = 13.02.24

\section{Automatische Security Updates}\label{sec:automatische-security-updates}
Aufgrund der sperrligen Besetzung der Technikabteilung der ~\gls{sv}, sowie der dauerhaften Aufgabenstellung alle ~\gls{it}-Systeme instand zu halten ist es ratsam automatische
Sicherheitsupdates auf Geräten zu konfigurieren.
Für solche Zwecke gibt es bereits vorgefertigte Programme, die einem Administrator dabei helfen. \\
Die in diesem Projekt verwendeten Technologien beschränken sich hierbei auf die Folgenden:
\begin{itemize}
    \item unattended-upgrades
    \item apt-listchanges
    \item apticron
\end{itemize}

%! Author = Charon
%! Date = 14/02/2024

\begin{lstlisting}[language=term,caption=Installation von Automatischen updates,label={lst:auto-updates-install}]
    $ sudo apt install unattended-upgrades apt-listchanges apticron
\end{lstlisting}
\linebreak
Zudem wurde sich bei der Konfiguration der automatischen Updates mithilfe einer bereits angefertigten Konfigurationsdatei beholfen.~\footcite{auto-update-config}
Diese Konfiguration muss zunächst in das Verzeichnis \textbf{/etc/apt/apt.conf.d/} kopiert werden.
Um zu verifizieren, ob die automatischen Updates funktionieren sollte der folgende Command eingegeben werden:
\begin{lstlisting}[language=term,caption=Verifikation der Konfiguration von automatischen Updates,label={lst:verify-auto-updates}]
    sudo unattended-upgrade -d --dry-run
\end{lstlisting}
\linebreak
Zum Nachverfolgen der automatisierten Updates werden die updates in \textbf{/car/log/unattended-upgrades/unattended-upgrades.log} geloggt.